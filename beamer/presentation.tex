%-------------------------------------------------------------------------------
%	PACKAGES AND THEMES
%-------------------------------------------------------------------------------

\documentclass[10pt]{beamer}

\mode<presentation> {

    \usetheme{default} %%%
    %\usetheme{Antibes} %%%
    %\usetheme{Berlin} %%%
    %\usetheme{Hannover} %%%
    %\usetheme{Luebeck} %%%
    %\usetheme{Malmoe} %%%
    %\usetheme{Pittsburgh} %%
    %\usetheme{Rochester} %%%
    %\usetheme{Singapore} %%%
    %\usetheme{Szeged} %%%

    %\usecolortheme{beaver}
    %\usecolortheme{beetle}
    %\usecolortheme{crane}
    %\usecolortheme{dolphin}
    %\usecolortheme{dove}
    %\usecolortheme{fly}
    %\usecolortheme{lily}
    %\usecolortheme{orchid}
    \usecolortheme{rose}
    %\usecolortheme{seagull}
    %\usecolortheme{seahorse}
    %\usecolortheme{whale}

    \usefonttheme[onlymath]{serif}

    % Remove the footer line in all slides
    %\setbeamertemplate{footline}

    % Replace the footer line in all slides with a simple slide count
    \setbeamertemplate{footline}[page number]

    % Remove the navigation symbols from the bottom of all slides
    \setbeamertemplate{navigation symbols}{}
}

\usepackage{fontspec}
\setsansfont{Liberation Sans}
\setmainfont{Liberation Serif}
\setmonofont[Scale=0.9]{Hack}
\usepackage{microtype} % Slightly tweak font spacing for aesthetics

\usepackage{graphicx} % Required for including pictures
\usepackage{wrapfig} % Allows in-line images such as the example fish picture
\usepackage[english]{babel} % Language hyphenation and typographical rules
\usepackage{hyperref}
\hypersetup{
    %draft, % Uncomment to remove all links (for printing in black and white)
    colorlinks = true,
    breaklinks = true,
    bookmarks  = true,
    bookmarksnumbered,
}
\usepackage{mathtools}
\usepackage{amsmath}
\usepackage{bookmark}
\bookmarksetup{numbered}
\usepackage{csquotes}
\usepackage{booktabs} % Horizontal rules in tables
\graphicspath{{pic/}} % Specifies the directory where pictures are stored

\usepackage[numbers,sort&compress]{natbib}
\bibliographystyle{acm}

%-------------------------------------------------------------------------------
%   CODE INCLUSION CONFIGURATION
%-------------------------------------------------------------------------------

\usepackage{listings}

\lstset{language=C,
        frame=LR,
        belowcaptionskip=1\baselineskip,
        breaklines=true,
        xleftmargin=\parindent,
        showstringspaces=false,
        basicstyle=\footnotesize\ttfamily,
        keywordstyle=\bfseries\color{Green},
        commentstyle=\color{Gray},
        identifierstyle=\color{Black},
        stringstyle=\color{Orange},
        %numbers=left, % Line numbers on left
        %firstnumber=1, % Line numbers start with line 1
        %numberstyle=\scriptsize\ttfamily\color{Brown},
}

%-------------------------------------------------------------------------------
%	TITLE PAGE
%-------------------------------------------------------------------------------

% The short title appears at the bottom of every slide, the full title is only
% on the title page
\title[Short title]{Presentation Title}

\author{Kuan-Yen Chou}
\institute[UIUC] {
    University of Illinois at Urbana-Champaign \\ % Institution full name
    \medskip
    \textit{kychou2@illinois.edu} % Your email address
}
\date{\today} % Date, can be changed to a custom date

\begin{document}

\begin{frame}
\titlepage % Print the title page as the first slide
\end{frame}

\begin{frame}
\frametitle{Overview} % Table of contents slide
\tableofcontents
\end{frame}

%-------------------------------------------------------------------------------
%	PRESENTATION SLIDES
%-------------------------------------------------------------------------------

\section{First Section}

\subsection{Subsection Example}

%------------------------------------------------

\begin{frame}
\frametitle{Paragraphs of Text}
Sed iaculis dapibus gravida. Morbi sed tortor erat, nec interdum arcu. Sed id
lorem lectus. Quisque viverra augue id sem ornare non aliquam nibh tristique.
Aenean in ligula nisl. Nulla sed tellus ipsum. Donec vestibulum ligula non lorem
vulputate fermentum accumsan neque mollis.\\~\\

Sed diam enim, sagittis nec condimentum sit amet, ullamcorper sit amet libero.
Aliquam vel dui orci, a porta odio. Nullam id suscipit ipsum. Aenean lobortis
commodo sem, ut commodo leo gravida vitae. Pellentesque vehicula ante iaculis
arcu pretium rutrum eget sit amet purus. Integer ornare nulla quis neque
ultrices lobortis. Vestibulum ultrices tincidunt libero, quis commodo erat
ullamcorper id.
\end{frame}

%------------------------------------------------

\begin{frame}
\frametitle{Bullet Points}
\begin{itemize}
\item Lorem ipsum dolor sit amet, consectetur adipiscing elit
\item Aliquam blandit faucibus nisi, sit amet dapibus enim tempus eu
\item Nulla commodo, erat quis gravida posuere, elit lacus lobortis est, quis
      porttitor odio mauris at libero
\item Nam cursus est eget velit posuere pellentesque
\item Vestibulum faucibus velit a augue condimentum quis convallis nulla gravida
\end{itemize}
\end{frame}

%------------------------------------------------

\begin{frame}
\frametitle{Blocks of Highlighted Text}
\begin{block}{Block 1}
Lorem ipsum dolor sit amet, consectetur adipiscing elit. Integer lectus nisl,
ultricies in feugiat rutrum, porttitor sit amet augue. Aliquam ut tortor mauris.
Sed volutpat ante purus, quis accumsan dolor.
\end{block}

\begin{block}{Block 2}
Pellentesque sed tellus purus. Class aptent taciti sociosqu ad litora torquent
per conubia nostra, per inceptos himenaeos. Vestibulum quis magna at risus
dictum tempor eu vitae velit.
\end{block}

\begin{block}{Block 3}
Suspendisse tincidunt sagittis gravida. Curabitur condimentum, enim sed
venenatis rutrum, ipsum neque consectetur orci, sed blandit justo nisi ac lacus.
\end{block}
\end{frame}

%------------------------------------------------

\begin{frame}
\frametitle{Multiple Columns}
\begin{columns}[c] % "c" centered vertical alignment; "t" top vertical alignment

\column{.45\textwidth} % Left column and width
\textbf{Heading}
\begin{enumerate}
\item Statement
\item Explanation
\item Example
\end{enumerate}

\column{.5\textwidth} % Right column and width
Lorem ipsum dolor sit amet, consectetur adipiscing elit. Integer lectus nisl,
ultricies in feugiat rutrum, porttitor sit amet augue. Aliquam ut tortor mauris.
Sed volutpat ante purus, quis accumsan dolor.

\end{columns}
\end{frame}

%------------------------------------------------
\section{Second Section}
%------------------------------------------------

\begin{frame}
\frametitle{Table}
\begin{table}
\begin{tabular}{l l l}
\toprule
\textbf{Treatments} & \textbf{Response 1} & \textbf{Response 2}\\
\midrule
Treatment 1 & 0.0003262 & 0.562 \\
Treatment 2 & 0.0015681 & 0.910 \\
Treatment 3 & 0.0009271 & 0.296 \\
\bottomrule
\end{tabular}
\caption{Table caption}
\end{table}
\end{frame}

%------------------------------------------------

\begin{frame}
\frametitle{Theorem}
\begin{theorem}[Mass--energy equivalence]
$E = mc^2$
\end{theorem}
\end{frame}

%------------------------------------------------

\begin{frame}[fragile] % Needed for verbatim
\frametitle{Verbatim}
\begin{example}[Theorem Slide Code]
\begin{verbatim}
\begin{frame}
\frametitle{Theorem}
\begin{theorem}[Mass--energy equivalence]
$E = mc^2$
\end{theorem}
\end{frame}\end{verbatim}
\end{example}
\end{frame}

%------------------------------------------------

\begin{frame}
\frametitle{Figure}
Uncomment the code on this slide to include your own image from the same
directory as the template .TeX file.
%\begin{figure}
%\includegraphics[width=0.8\linewidth]{test}
%\end{figure}
\end{frame}

%------------------------------------------------

\begin{frame}[fragile] % Needed for verbatim
\frametitle{Citation}
An example of the \verb|\cite| command to cite within the presentation:\\~

This statement requires citation \cite{figueredo2009}.
\end{frame}

%------------------------------------------------

\begin{frame}
\frametitle{References}
\footnotesize{
    \setsansfont{Liberation Serif}
    \bibliography{ref}
}
\end{frame}

%-------------------------------------------------------------------------------

\end{document}
